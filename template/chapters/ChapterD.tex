% !Mode:: "TeX:UTF-8"

\chapter{分层光网络下的并行路由优化算法研究}
\section{引言}
  随着视频流、云计算服务和移动应用的普及,互联网的流量不断增加[],业务量的持续增长给互联网的传输带来了巨大的挑战,为了满足日益增长的容量要求,波分复用(WDM)系统已部署在骨干网络,其每个通道带宽高达40 GB/s或100 Gb/s。现在,400 Gb/s的带宽接口已经在实验室[2]实现,此外,为了满足不同的数据速率服务异构的流量类型,混合线路速率(10/40/100Gb/s)的WDM系统也已经部署。因此,WDM光网络视乎为大流量业务传输提供了一个既实际又高效的解决方案。
  然而,传统的WDM光网络严格遵循ITU-T的固定均匀间距和网格(通常为50)GHz或100千兆赫)[],这样会导致低效的频谱利用率,比如,一个较大的波长可能会被分配给一个低速率的业务,即使这个业务根本就不能占满整个波长。很明显,传统WDM光网络的不灵活和粗粒度的带宽控制会导致显著的频谱浪费,限制了提高其网络容量的潜力。
  为了应对WDM网络的低敏捷性和频谱浪费问题,近年来,弹性光网络(EON)的架构被提出,在EON架构中,存在细粒度的带宽间隙(比如,12.5GHZ),他比WDM网络所遵循的ITU标志的50GHZ或者100GHZ的带宽粒度要小很多,而且,这些带宽间隙可以根据需要被组合在一起以提供更宽的通道。所以为了提高带宽利用率,在EON中存在混合速率,每个速率的业务需要不同数量的频谱间隙数量。理论上,EONs能够灵活地提供各种速率,但是在实际中,EONs却可能只含有很少的速率种类,这主要是因为:第一,随着频谱的增加,频谱碎片和管理复杂度将显著增加。第二,实际的EON从较低的线路费率升级到更高的线路费率,并存大量速率的情况已经很少见了。
  为了在EON网络中加入业务,控制平面必须在网络中找到一条路径,同时,还需要在此路径上的链路上分配足够的频谱带宽,来创建一个合适的端到端光路连接,这被称为路由和频谱分配(RSA)问题,EONs中的RSA问题比传统的在WDM网络中的路由和波长分配(RWA)问题更具有挑战性,传统的RWA解决方案[][]在EONs中已经变得不再适应了。在EONs中一个业务需求可能需要多个连续的频谱间隙,而且由于缺少全光谱转换器,每条光连接的频谱从它的源节点到它的目的节点 ,在所经过的链路上保持不变,这被称为频谱连续性限制。
  根据流量模式,RSA问题可以进一步分为静态RSA问题和动态RSA问题。静态的RSA问题出现在网络规划阶段,其中流量需求是已知的,这样可以离线计算出最优或者接近最优的RSA解,动态RSA问题是指在实时业务情况下光通路的路由选择和波长分配的优化问题,在动态RSA问题中,业务随机的到达和离开光网络,而且当业务到达网络时,控制平面需要在短时间内找到RSA解来安排业务。动态RSA问题比静态RSA问题更具有挑战性,因为业务需求随机到达和离开,网络状况随时发生变化,而且要求控制平面反应实时。
 链路代价表示路由路径选择这一条链路所需要的代价,链路代价可以设置为链路的距离,延迟,租用链路需要的费用等等,业务的路由代价表示业务路径经过的链路的代价总和,路由代价反应了路由路径的优劣程度,如果链路代价设置为延迟,则路由代价越小表示路由整体延迟较小,如果链路代价设置为1,则路由代价越小,表示路径经过的跳数越小,使用的链路资源越小。
 本章考虑在EONs中解决动态RSA问题,设计出带跳限约束下的基于分层图的动态路由优化算法来优化整体路由代价,分别考虑无权图和带权图上的算法GPU加速设计,GPU并行版本达到平均近6倍的算法加速比,实验与分层图上的传统算法比较,显示该算法过程在短时间内能够大量的优化路由代价,并且由于路由选择较优,使用网络资源较少,最终网络阻塞率表现也有显著的提高,实现了快速的路由代价优化和整体的阻塞避免。
\section{问题描述}
\subsection{EONs中动态RSA问题}
  在动态场景中,业务的到达时间和服务时间都是随机的,当业务到达网络时,SDN控制层需要找到可行的路径,并且为路径分配合适的频谱资源。由于EON的物理限制,RSA问题的解需要满足以下限制:
\begin{enumerate}[(i)]
\item 传输距离限制:光信号的传输质量会随着传输距离的增加而降低,为了在目的节点上成功恢复光信号。为了简化设计,本文假设每条链路的距离相同,所以光网络中的链路跳数需要小于一个阈值$D_max$。
\item 频谱连续性限制:每条光连接的频谱从它的源节点到它的目的节点 ,在所经过的链路上保持不变。
\item 频谱不重叠限制:同一光纤链路中的频谱不能分配给不同的光路。
\item 频谱邻近限制:频谱邻近约束保证分配给一个光路径的频谱必须是一个连续的部分。
\end{enumerate}
\section{主要优化流程}
\section{无权图情况下的GPU算法设计}
\section{带权图情况下的GPU算法设计}
\section{实验仿真分析}